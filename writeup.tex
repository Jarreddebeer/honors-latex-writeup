\documentclass[prodmode,acmtecs]{acmsmall} % Aptara syntax

% Package to generate and customize Algorithm as per ACM style
\usepackage[ruled]{algorithm2e}
\renewcommand{\algorithmcfname}{ALGORITHM}
\SetAlFnt{\small}
\SetAlCapFnt{\small}
\SetAlCapNameFnt{\small}
\SetAlCapHSkip{0pt}
\IncMargin{-\parindent}



% DOI
\doi{0000001.0000001}

% Document starts
\begin{document}

% Title portion
\title{Accelerating automated extraction of radio astronomical sources from observation data with GPU accelerators}
\author{by Jarred de Beer}

\begin{abstract}
This document categorises various papers relating to identifying sources in datacubes.
The intention is to select a source finding algorithm which can be optimised and rewritten for
high performance GPU processing and maybe also parallelized on a cluster. A potential goal is that
it can be run in production on HI surveys such as MEERKAT, helping the feasibility 
in processing large volumes of data.
\end{abstract}

\maketitle

% Head 1
\section{Source finding Algorithms}

A description and comparison of source finding algorithms is listed in \cite{popping2012comparison}.

Most of these algorithms are based on Intensity thresholds,
which observe every point in a datacube for sources. 
SExtractor, SFind and Duchamp are good examples of these. 
They all share an inherent limitation (see \cite{jurek2012characterised} for details).

Alternatively, The Characterised Noise HI (CNHI), uses a different approach and does not suffer
from this limitation. 
I'd like to explore CNHI as the algorithm to parallelize, and will
keep it in mind when listing the different papers. I'm holding thumbs Yaseen hasn't picked it out
as well, otherwise I might need to refocus.

\subsection{Sofia}

SoFiA \cite{serra2015sofia} puts different source-finding algorithms together in a single piece
of software. A summary of these are listed in \cite{koribalski2012source}.

When merging sources SoFiA does not take into account their individual sizes, instead, merging
is parameterised by the user.
More specialised source finders are able to take sizes into account, such as Clumpfield \cite{williams1994determining}
and BLOBCAT \cite{hales2012blobcat}.

BLOBCAT \cite{hales2012blobcat} uses the flood fill algorithm to detect and catalogue blobs of pixels
as sources.

Sources are identified with the C++ implementation of the Lutz one pass algorithm \cite{lutz1980algorithm}
by \cite{jurek2012characterised}. The Lutz one pass algorithm is also implemented in the Characterised
noise HI source finder (CNHI) listed below, which we will focus on.

\subsection{Scalable Source Finding Framework (SSoFF)}

SSoFF details a method for constructing parallel source finders which can then make use of HPC architecture.
Particular libraries and API's need to be used: MPI, MPI-IO and OpenMP. Converting serial programs
may take significant effort.

\subsection{DUCHAMP source finder}

DUCHAMP \cite{whiting2012duchamp} is an Intensity threshold source finder.
The noise suppression of DUCHAMP has been studied for multi-threading,
SIMD, and memory-management on desktop computers \cite{scottBadenhorst}. The Selavy project
\cite{whiting2012source} is a parallelized implementation of DUCHAMP, and might be useful.

In the context of CPU processing, Scott's paper \cite{scottBadenhorst} notes that the combination of SIMD and hyperthreading
is terrible for performance. However, in the context of GPU processing, \cite{mahesri2008tradeoffs}
finds that MIMD has superior performance to SIMD. This is useful since we're considering GPU processing.


\subsection{Gaussian fitting algorithms}

Gaussian fitting algorithms \cite{hancock2012compact} are related to Selavy and of possible interest.

\subsection{Monte Carlo method}

The Monte Carlo method \cite{allison2011application} is a Bayesian model fitting.
This is an extension to Gaussian models. It does not use a 3D source finder
such as DUCHAMP, and it may be relevant to CNHI.

\subsection{Multi Scale Multi Frequency Synthesis Algorithm}

Also related to Selavy, the Multi Scale Multi Frequency Synthesis Algorithm will be used in
the imaging pipeline of the Australian Square Kilometre Array Pathfinder (ASKAP) \cite{rau2009advances}.

\subsection{CNHI source finder}

The Characterised noise HI source finder (CNHI) \cite{jurek2012characterised} 
is an alternative to Intensity based source finders.

CNHI uses Matched filtering, which is known as the best method for automatically detecting sources.
Like SoFiA, it implements the Lutz one-pass algorithm with a novel, 
sparse representation that has high compression.

A limitation of Intensity based source finders, which CNHI does not suffer from,
is that they struggle to detect sources when the flux is spread over extra voxels.
CNHI has its own limitations though.

There are three core concepts behind the CNHI algorithm:

\begin{itemize}
\item Datacubes can be treated as a bundle of HI spectra (ideal for parallelization), as opposed to a collection of voxels.
\item Contiguous blocks of voxels, rather than individual sources, should be tested for sources.
\item Sources are detected where regions do not appear as noise (the inverse to shape based algorithms)
\end{itemize}

Noise is detected with Comparative statistical tests:
the Kolmogorov-Smirnov test \cite{kendall1979advanced} and the Kuiper test \cite{kuiper1960tests}.

CNHI's completeness, reliability, and performance is presented in 
\cite{popping2012comparison}, with a complementary comparison discussed in \cite{westmeier2012basic}. 
Its terminology is defined and measured in \cite{popping2012comparison}.

\subsection{The Gamma-Finder}

The Gamma-Finder \cite{boyce2003gammafinder} estimates noise variance, known as the Gamma Statistic.
A signal-to-noise ratio is then used to clip the data. There are no parameters to the Gamma-Finder.

\subsection{2D-1D Wavelet Reconstruction finder}

2D-1D Wavelet Reconstruction \cite{floer20122d} performs a wavelet transform in all planes
of the cube and a 1D wavelet transform at each pixel.

\subsection{Smooth plus Clip source finder}

This algorithm optimises the signal-to-noise ratio of objects in a cube. It looks for sources
in both the original and the smoothed cube. Smoothing can be done on the sky, velocity, or 
all three axes. \cite{serra2012atlas3d}

\subsection{Parallel Gaussian Source Finder}

The Parallel gaussian Source Finder is built on SSoFF \cite{westerlund2014framework} for parallelism.

\subsection{GPU-accelerated Filter-based source finder}

So far i've only identified \cite{westerlund2015performance} as having ported a source finding
algorithm onto the GPU, so this paper will also be looked at closely.

It states that GPUs are a good fit for datacubes, but that separate effort is needed to port any source
finding algorithm onto the GPU. This effort implies that porting an algorithm to the GPU 
would be of direct benefit to the community, and is why we are choosing to do so.

The filtering section takes up to 70\% of the running time, while other time
consuming tasks are statistics calculations and reading image and weights data.

\cite{westerlund2015performance} Parallelizes a filtering algorithm based on CUDA, OpenCL,
OpenMP and OpenMPI. It lists the following as potential future work, which could form areas 
of meaningful contribution:

\begin{itemize}
\item Porting the statistics functions to GPU as they contain a significant portion of run time.
\item Adapting the program to process part of a nodes image data while the rest is being read in.
\item Modify the current implementation to perform additional work with less communication.
\item Add automatic configuration for best thread topology and performance.
\item Port any other algorithms added to SSoFF onto the GPU.
\end{itemize}


\subsection{Additional GPU considerations}

In the paper \cite{sung2014place} an algorithm for the in-place matrix transposition
on GPUs is discussed. DUCHAMP uses transposition to keep the memory accesses coherent in the denoising
process and may also be useful for the contiguous blocks of voxels searched by CNHI.

\bibliographystyle{ACM-Reference-Format-Journals}
\bibliography{writeup}


\end{document}


